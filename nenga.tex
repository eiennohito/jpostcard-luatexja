% -*- program: lualatex -*-

\documentclass{jlpostcard}

\usepackage[hiragino-pro]{luatexja-preset}

% Sender's information
\sendername{豊臣&秀吉\\&ねね}
\senderaddressa{大阪府大阪市中央区}
\senderaddressb{大阪城1−1}
\senderpostcode{5400002}

\begin{document}

\addaddress
  {織田&信長&殿\\&濃姫&様}
  {5211341}
  {滋賀県近江八幡市安土町下豊浦}    
  {安土城天主}

\end{document}